\chapter{I 级叶/盘协调转子固有振动特性分析}
\section{基础知识}
\subsection{有限元法}
\subsection{循环对称结构的分析方法}
\section{I 级叶/盘转子振动特性的有限元分析}
\subsection{计算模型}

\subsection{有限元计算结果及分析}

\chapter{I级叶/盘转子错频方案的对比分析}

在叶轮机械领域,对一个实际的叶盘转子,错频是指由于单个叶片之间因几何上
或结构上的不同而造成的其在固有频率上的差异\cite{图书馆史研究}。\par

......\par

\section{多自由度系统的强迫响应分析}

由前面的分析可知,响应分析在数学上是一个具有 38 个自由度的二阶线性微分方
程的数值积分问题\cite{中国学术期刊标准化数据库系统工程的,maskrcnn}。\par

\subsection{动态响应的计算方法}

\begin{enumerate}
\item 系统的运动方程\\
多自由度系统运动微分方程的一般形式为:
\begin{enumerate}
\item $\cdots$
\item $\cdots$
\end{enumerate}
\item 微分方程组的数值积分\\
一阶常系数微分方程组的初值问题可表述为: ......
\end{enumerate}

\subsection{强迫响应分析前的准备工作}
......\par

