\documentclass[openright, twoside]{gdutthesis-excellent}

\usepackage{zhlipsum}

\let\cleardoublepage\relax

\begin{document}
\renewcommand{\baselinestretch}{1.5}
{\centering
\vspace*{20pt}
\zihao{2}\bfseries
基于Jetson开发平台的实例分割系统设计与实现\par
\vspace*{16pt}
\normalfont \zihao{5}\kaishu
%  \vspace{0.5\baselineskip}
  戴逸伦\footnote{\zihao{6}戴逸伦(1997—),男,物联网工程专业2015届毕业生,工学学士。} \quad\quad 指导老师:李东\\
%  \vspace{0.5\baselineskip}
  (广东工业大学自动化学院,广州,510090)\\
%  \vspace{0.5\baselineskip}

}

\renewcommand{\baselinestretch}{1.46}
\begin{flushleft}
\setlength\leftskip{2em}
\setlength\rightskip{2em}
\setlength{\parindent}{4em}
\kaishu\zihao{-5}
{\quad\quad\heiticu 摘\quad 要}文主要介绍小区智能化的基础是其通信网络, ......\par
{\zihao{-5}\heiticu 关键词:}智能小区 网络\par
{\zihao{-5}\heiticu 评阅人评语:}该方案对智能小区的信息网络系统作了全面的设计规划。......是一个完整、可行、实用、先进的方案。(评阅人:万频副教授)\par
\end{flushleft}

\chapter{I 级叶/盘协调转子固有振动特性分析}
\section{基础知识}
\subsection{有限元法}
\subsection{循环对称结构的分析方法}
\section{I 级叶/盘转子振动特性的有限元分析}
\subsection{计算模型}

\subsection{有限元计算结果及分析}

\chapter{I级叶/盘转子错频方案的对比分析}

在叶轮机械领域,对一个实际的叶盘转子,错频是指由于单个叶片之间因几何上
或结构上的不同而造成的其在固有频率上的差异\cite{图书馆史研究}。\par

......\par

\section{多自由度系统的强迫响应分析}

由前面的分析可知,响应分析在数学上是一个具有 38 个自由度的二阶线性微分方
程的数值积分问题\cite{中国学术期刊标准化数据库系统工程的,maskrcnn}。\par

\subsection{动态响应的计算方法}

\begin{enumerate}
\item 系统的运动方程\\
多自由度系统运动微分方程的一般形式为:
\begin{enumerate}
\item $\cdots$
\item $\cdots$
\end{enumerate}
\item 微分方程组的数值积分\\
一阶常系数微分方程组的初值问题可表述为: ......
\end{enumerate}

\subsection{强迫响应分析前的准备工作}
......\par



\setlength{\baselineskip}{10pt}
\renewcommand{\baselinestretch}{1}
\begin{center}
\zihao{5}\heiticu 参考文献
\end{center}

\renewcommand{\baselinestretch}{1.2}
\zihao{5}\kaishu
\setlength{\parindent}{0em}
\zihao{-5}
1 张振昭,许锦标,万频主编.楼宇智能化技术.北京:机械工业出版社,1999\par
1 张振昭,许锦标,万频主编.楼宇智能化技术.北京:机械工业出版社,1999\par

\vspace{36pt}

\renewcommand{\baselinestretch}{1}
{
\centering
\zihao{4}\textbf{The design of the intelligent small area network system}\\
\zihao{4} Qile Wang\\
}
\zihao{5}
\textbf{Abstract:}The article mostly introduced that the base of the intelligent small area New is it's communication network, and expounded the design of the intelligent small ......\par
\textbf{Key words:}Intelligent small area Network,HFC\par



\end{document}
