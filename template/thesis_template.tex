\documentclass[doctor, openright, twoside]{gdutthesis}
\usepackage{lipsum} % 用于生成英文假文
\usepackage{zhlipsum} % 用于生成中文假文

\begin{document}

% 生成封面
\maketitle

\frontmatter % 正文前的摘要部分(罗马数字页码)

% 生成目录
\tableofcontents

% 生成书脊
\spine

%%%%%%%%%%
% 摘要
%%%%%%%%%%
\begin{abstract}{chinese}

\end{abstract}

\begin{abstract}{english}

\end{abstract}

\mainmatter % 正文开始(阿拉伯数字页码)

% 诸论

% 正文

% 结论

\printbibliography[heading=bibintoc] % 参考文献

\begin{appendix}
\appendixpage% 插入“附录”字样的分割页

\chapter{附录1标题}

\chapter{附录2标题}

\end{appendix}


%%%%%%%%%%
% 附录
%%%%%%%%%%

\begin{appendix}
\appendixpage % 插入“附录”字样的分割页

\end{appendix}

\backmatter % 文后无编号部分

% 致谢
\begin{thanks}

\end{thanks}

\end{document}
